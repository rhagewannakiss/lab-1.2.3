\documentclass[12pt]{article}
\usepackage[utf8]{inputenc}
\usepackage[russian,english]{babel}
\usepackage[T2A]{fontenc}
\usepackage[left=10mm, top=20mm, right=18mm, bottom=15mm, footskip=10mm]{geometry}
\usepackage{indentfirst}
\usepackage{amsmath,amssymb}
\usepackage[italicdiff]{physics}
\usepackage{graphicx}
\graphicspath{{images/}}
\DeclareGraphicsExtensions{.pdf,.png,.jpg}
\usepackage{wrapfig}



\usepackage{caption}
\captionsetup[figure]{name=Рисунок}
\captionsetup[table]{name=Таблица}
  
\title{Лабораторная работа 1.2.3\\ \\ \\Определение моментов инерции твердых тел с помощью трифилярного подвеса}
\author{Макарская Александра, Б01-401\\МФТИ ФРКТ}
\date{9 октября 2024 г.}



\begin{document}
\maketitle


\subsection*{Доказательство аддитивности инерции}
    Используя $I = kmT^2$ получаем рассчитываем моменты инерции для тел на трифилярном подвесе:

\begin{table}[h!]
\begin{center}
\begin{tabular}{|c|c|c|c|c|c|}
\hline
Тела               & Период & $I_{\Sigma}$ экспер., $\text{кг}\cdot\text{м}^2$ & Табличный & $I_{\Sigma}$ расчет., $\text{кг}\cdot\text{м}^2$ & Точность \\ \hline
Платформа          & 4,415  & 0,00765                                  & --        & --                               &  --        \\ \hline
Диск+Платформа     & 3,95   & 0,00989                                  & 0,00216   & 0,00981                          & $0,82\%$ \\ \hline
Цил+Платформа      & 4,24   & 0,0143                                   & 0,00686   & 0,0145                           & $1,38\%$ \\ \hline
Стерж+Платформа    & 3,756  & 0,01169                                  & 0,00383   & 0,01148                          & $1,74\%$ \\ \hline
\end{tabular}
\end{center}
\end{table}

Сравнивая моменты инерции полученные расчетно и экспериментально, убеждаемся что с высокой точностью в $\pm 1\%$  верна аддитивность моментов инерции!

\subsection*{Теорема Гюйгенса-Штейнера}

Построим график $I(h^2)$:




С использованием аппроксимации получаем, что $I = k h^2 + b$ , где k = 1.043 кг и b = 10,57 $\text{кг}\cdot\text{м}^2$.
При этом m = 1.05 кг установки также совпадает с k, а $I(0)$ с погрешностью 0,48\% совпадает с b.

\[k=\frac{\langle xy\rangle-\langle x\rangle \langle y\rangle}{\langle x^2\rangle - \langle x\rangle^2}\approx 1.0426 кг\]

\[\sigma_{k} = \frac{1}{\sqrt{N}}\sqrt{\frac{\langle y^2 \rangle - \langle y \rangle ^2}{\langle x^2 \rangle - \langle x \rangle ^2} - k^2} \approx 0,015 (1,5\%)\]

\[\text{Итого: } k = 1,0426 \pm 0,015 \text{ кг}\]


		\item Определим необходимое количество колебаний для измерений периода с точностью $\varepsilon_T = 0.5\%$.
		\[N = \dfrac{\sigma_T}{T\varepsilon_T} < 1\]
		Для надежности возьмем $N = 5$, так как кроме систематической, измерения могут содержать случайную погрешность (особенности счетчика).

		\item Измерим параметры установки $l$, $R$ и $r$, $m$ их погрешности (табл.3). Найдем $z_0$ по формуле:
		\[z_0 = \sqrt{l^2 - R^2} = \sqrt{2154^2 - 114.6^2} = (2151 \pm 2) \text{ мм}\]
		
		\begin{table}[h]
			\centering
			\caption{Параметры установки}
			\begin{tabular}{|l|c|c|c|}
				\hline
				& Величина & $\sigma$ & $\varepsilon$ \\
				\hline
				$l$, мм & 2154.0 & 2 & 0.0009 \\
				\hline
				$z_0$, мм & 2151.0 & 2 & 0.0009 \\
				\hline
				$r$, мм & 30.5 & 0.3 & 0.0098 \\
				\hline
				$m$, г & 934.7 & 0.5 & 0.0005 \\
				\hline
			\end{tabular}
		\end{table}
		
		Вычислим константу $k$ для данной установки и ее погрешность:
		\[k = \dfrac{gRr}{4\pi^2z_0} \approx 0.404 \cdot 10^{-3} \frac{\text{м}^2}{c^2}\]
		\[\varepsilon_k = \varepsilon_g + \varepsilon_R + \varepsilon_r + \varepsilon_{z_0} \approx 0.0111\]
		\[\sigma_k = k\varepsilon_k \approx 0.037 \cdot 10^{-3} \frac{\text{м}^2}{c^2}\]
		
		\[k = \left(0.40 \pm 0.04\right) \cdot 10^{-3} \frac{\text{м}^2}{c^2}\]
		
		\item Опрелелим момент инерции ненагруженной платформы $I_0$:
		\[I_0 = kmT^2 \approx 7.11 \cdot 10^{-3} \text{ кг}\cdot\text{м}^2\]
		\[\varepsilon_{I_0} = \varepsilon_k + \varepsilon_m + 2\varepsilon_T \approx 0.0216\]
		\[\sigma_{I_0} = I_0\varepsilon_{I_0} \approx 0.154 \cdot 10^{-3} \text{ кг}\cdot\text{м}^2\]
		
		\[I_0 = (7.11 \pm 0.15) \cdot 10^{-3} \text{ кг}\cdot\text{м}^2\]
		
		\item Измерим параметры имеющихся тел:
		
		\begin{table}[h]
			\centering
			\caption{Параметры тел}
			\begin{tabular}{|c|c|c|c|c|c|c|}
				\hline
				№ & Схема & Параметры & T, c & $I + I_0, 10^{-3} \text{ кг}\cdot\text{м}^2$ & $I, 10^{-3} \text{ кг}\cdot\text{м}^2$ & $I_\text{теор}, 10^{-3} \text{ кг}\cdot\text{м}^2$\\
				\hline
				1 & \rowincludegraphics{body1} &
				\begin{tabular}{c}
					$h = (55.4 \pm 0.1) \text{ мм}$ \\
					$d = (3.9 \pm 0.1) \text{ мм}$ \\
					$D = (158.5 \pm 0.1) \text{ мм}$ \\
					$m$ = 748.0 г \\
				\end{tabular} & 4.150 & 11.59 & $4.48 \pm 0.25$ & 4.58 \\
				\hline
				2 & \rowincludegraphics{body2} &
				\begin{tabular}{c}
					$a = (26.9 \pm 0.1) \text{ мм}$ \\
					$b = (26.9 \pm 0.1) \text{ мм}$ \\
					$c = (208.5 \pm 0.1) \text{ мм}$ \\
					$m$ = 1177.5 г \\
				\end{tabular} & 3.69 & 11.56 & $4.45 \pm 0.25$ & 4.33 \\
				\hline
				3 & \rowincludegraphics{body3} &
				\begin{tabular}{c}
					$d = (20.0 \pm 0.1) \text{ мм}$ \\
					$D = (158.5 \pm 0.1) \text{ мм}$ \\
					$h = (7.0 \pm 0.1) \text{ мм}$ \\
					$H = (30.5 \pm 0.1) \text{ мм}$ \\
					$m$ = 1122.9 г \\
				\end{tabular} & 3.584 & 10.57 & $3.46 \pm 0.23$ & 3.30\\
				\hline
				1 + 3 & & m = 1870.9 г & 3.664 & 15.07 & $7.96 \pm 0.33$ & 7.88 \\
				\hline
			\end{tabular}
		\end{table}
		
		Расчитаем теоретические значения моментов инерции тел и запишем в табл. 4:
		\[I_1 = \frac{1}{2}m\left(r_1^2+r_2^2\right) = \frac{1}{2}m\left(\left(\frac{D - d}{2}\right)^2 + \left(\frac{D}{2}\right)^2\right) = 4.58 \cdot 10^{-3} \text{ кг}\cdot\text{м}^2\]
		\[I_2 = \frac{1}{12}m\left(a^2+c^2\right) =  4.33 \cdot 10^{-3} \text{ кг}\cdot\text{м}^2\]
		\newpage
		\[m_1 = m\dfrac{V_1}{V} = m\dfrac{d^2H}{d^2H+D^2h}\]
		\[m_2 = m\dfrac{V_2}{V} = m\dfrac{D^2h}{d^2H+D^2h}\]
		\[I_3 = \frac{1}{8}m_1d^2 + \frac{1}{8}m_2D^2 = \dfrac{1}{8}m\dfrac{d^4H + D^4h}{d^2H+D^2h} = 3.30 \cdot 10^{-3} \text{ кг}\cdot\text{м}^2\] 
		\item Измерим моменты инерций всех тел и запишем в табл. 4. Момент инерции и его погрешность расчитаем по формуле:
		\[I = k(m_0+m)T^2 - I_0\]
		\[\sigma_I = \sigma_{I_0} + \sigma_{I} = \varepsilon_{I_0}I_0 + \varepsilon_{I_0}I = \varepsilon_{I_0} (I_0 + I)\]
		
		Как видим, все измеренные моменты инерции $I_i$ не выходят за пределы погрешности $\sigma_{I_i}$.
		
		\item Измерим момент инерции тел 1 и 3 вместе, результаты запишем в табл. 4. Как видно из таблицы, аддитивность моментов инерции соблюдается, значение лежит в пределе допустимой погрешности:
		\[I_{1+3} = (7.96 \pm 0.33) \cdot 10^{-3} \text{ кг}\cdot\text{м}^2\]
		\[I_1 + I_3 = (7.94 \pm 0.48) \cdot 10^{-3} \text{ кг}\cdot\text{м}^2\]
		
		\item Поместим на платформу диск, разрезанный по диаметру, горизонтально. Постепенно раздвигая половинки диска так, чтобы их общий центр масс все время оставался на оси вращения платформы, снимем зависимость момента инерции системы $I$ от расстояния $h$ каждой из половинок до центра платформы. Масса грузиков $m$ = 1.336 кг. Расчитаем моменты инерции по формуле и запишем в табл. 5:
		\[I = k(m+m_0)T^2 - I_0\]
		\begin{table}[h]
			\centering
			\caption{Сдвиг половинок цилиндра}
			\begin{tabular}{|c|c||c|c||c|c|c|}
				\hline
				№ & $h$, мм & $T$, c & $I, 10^{-3} \text{ кг}\cdot\text{м}^2$ & $T$, c & $I, 10^{-3} \text{ кг}\cdot\text{м}^2$ \\
				\hline
				1  & 0  & 3.094 & 1.58 & 3.012 & 1.13 \\
				2  & 5  & 3.098 & 1.61 & 3.020 & 1.17 \\
				3  & 10 & 3.116 & 1.71 & 3.040 & 1.29 \\
				4  & 15 & 3.142 & 1.86 & 3.068 & 1.44 \\
				5  & 20 & 3.188 & 2.12 & 3.104 & 1.64 \\
				6  & 25 & 3.226 & 2.34 & 3.164 & 1.98 \\
				7  & 30 & 3.294 & 2.75 & 3.222 & 2.32 \\
				8  & 35 & 3.370 & 3.21 & 3.298 & 2.77 \\
				9  & 40 & 3.444 & 3.66 & 3.382 & 3.28 \\
				10 & 45 & 3.550 & 4.34 & 3.466 & 3.80 \\
				11 & 50 & 3.634 & 4.88 & 3.562 & 4.41 \\
				\hline
			\end{tabular}
		\end{table}
		
		Построим график зависимости $I(h^2)$. По графику видно, что он представляет собой линейную зависимость $I = kh^2 + b$.
		
		По формуле Гюйгенса-Штейнера:
		
		\[I(h) = I + mh^2\]
		
		Найдем коэффициенты по МНК:
		
		\[I = b = (1.565 \pm 0.009) \cdot 10^{-3} \text{ кг}\cdot\text{м}^2\]
		\[m = k = (1.335 \pm 0.011) \text{ кг}\]
		
		\item Повторим измерения для вертикального положения половинок, запишем в табл. 5.
		
		\begin{figure}[h]
			\centering
			\caption{Графики зависимостей $I(h^2)$ для разных положений половинок}
			\includegraphics{graph}
		\end{figure}
		
		Найдем коэффициенты по МНК:
		
		\[I = b = (1.142 \pm 0.005) \cdot 10^{-3} \text{ кг}\cdot\text{м}^2\]
		\[m = k = (1.336 \pm 0.006) \text{ кг}\]
		
		Как видно из эксперимента, формула Гюйгенса-Штейнера работает, а массы цилиндра, вычисленные по МНК, лежат в пределах допустимой погрешности.

\end{document}
